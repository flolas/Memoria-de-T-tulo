%\addcontentsline{toc}{chapter}{Resumen Ejecutivo}
    \begin{center}
        \vspace*{1cm}
        
        \textbf{Resumen Ejecutivo}
        
        \vspace{0.5cm}
        
    \end{center}



La educación, tanto en Chile como en el mundo, es un tema muy importante ya que es la base para el desarrollo de una sociedad de calidad. Sin embargo, existe un grupo de estudiantes que se ve enfrentado a la decisión de desertar de el sistema educativo. Es por esto que la finalidad de este documento es la creación de un prototipo de sistema de alerta temprana que permita identificar en tiempo real aquellos estudiantes con riesgo de desertar del sistema escolar o algún foco de deserción escolar, y gracias a esto apoyar la toma de desiciones en cuanto a políticas públicas dirigidas a la educación pública del país.

Siguiendo la metodología CRISP-DM, utilizamos herramientas de minería de datos como árboles de decisiones, bosques aleatorios y redes neuronales entre otras, eligiendo la que más se acerque al objetivo de modelo. Para poder hacer esta selección del modelo de predicción se utilizaron procedimientos de múltiples validaciones cruzadas y algoritmos que optimizan los parámetros de los modelos para obtener soluciones más cercanas a la solución óptima.

Logramos crear un modelo acorde a nuestros objetivos capaz de predecir la deserción escolar para el año siguiente con una precisión del $89$\% basado en un meta-algoritmo llamado AdaBoost. Nuestra muestra de estudio consistió en $260.000$ alumnos de escuelas municipalizadas y subvencionadas del sistema regular de educación del Gran Santiago de Santiago de Chile con un total de $120$ características donde finalmente solamente $10$ eran importantes:la diferencia de edad entre los pares del alumno, el ranking del promedio general y asistencia en el curso del alumno, la escolaridad de los padres, el grupo socioeconómico del barrio del establecimiento al que asiste el alumno y del lugar donde reside entre otros. 

Finalmente se concluye que si bien el modelo es capaz de predecir la deserción escolar al plazo de un año, pero no es capaz de explicar la deserción. El problema en particular es que no es capaz de detectar casos de deserción en periodos menores a un año debido a la baja frecuencia y disponibilidad de los datos y la escasa información personal del alumno que dificultan la predicción de la deserción escolar.Para el futuro se espera obtener una mayor frecuencia y disponibilidad de datos, lo que posibilitaría un sistema de predicción de desertores para monitorear de forma semestral.
