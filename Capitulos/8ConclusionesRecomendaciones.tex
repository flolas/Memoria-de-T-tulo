\chapter{Conclusiones y Recomendaciones}
\label{ch:concureco}

Durante este trabajo hemos desarrollado un sistema de alerta temprana de la deserción escolar utilizando varias fuentes de información. Creamos el SPDE con la finalidad de poder tener una herramienta predictiva para detectar la deserción de forma temprana y así los tomadores de decisiones puedan fundamentar y medir el impacto de sus intervenciones y decisiones.
Este trabajo muestra que un sistema de predicción utilizando variables públicas como el promedio y la asistencia del alumno es bastante certero, esto se explicó por que básicamente estas variables resumen la mayor parte de la información del alumno. Un nuevo estudio, pensando en una herramienta explicativa, sería recolectar más información desagregada como lo hemos comentado, pero lamentablemente esto no es viable en el día de hoy.

Las recomendaciones que se desprenden del trabajo son las siguientes:
\begin{itemize}
\item El MINEDUC debe mejorar la calidad de los datos, puesto que durante el trabajo se encontraron errores humanos en la codificación, formato y el estándar. Se evidenció que cada año tenía un diferente estándar en términos de codificación e incluso, en algunos años habían problemas con alumnos repetidos.
\item Además, el MINEDUC debe establecer un mejor marco de trabajo para recolectar la información, puesto que, los datos se liberan, generalmente, de forma anual imposibilitando el uso de un modelo de predicción de alerta temprana más realista donde capte la información que ocurre de un mes a otro.
\item La agencia de calidad debe mejorar la información generada del SIMCE y las encuesta. Parte de nuestros esfuerzos se enfocaron en reparar la encuesta SIMCE, puesto que, como lo hemos comentado durante el trabajo, estas tenían alumnos repetidos(el identificador único) con diferente información.
\end{itemize}