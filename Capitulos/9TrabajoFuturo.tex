\chapter{Trabajo Futuro}
\label{ch:trafu}
Acá hay que poner énfasis en que en la metodología de CRISP-DM realizamos todo menos la implementación y que esta normalmente se realiza junto al cliente. Señalar la creación de una página web y un modelo persistente 

El presente trabajo se enfocó en la creación de un \textbf{prototipo} de lo que debiese ser un sistema de alerta temprana para detectar la deserción. Durante el camino Se encontraron diferentes problemas, por ejemplo, las variables no son capaces de explicar la deserción, pero sí predecir con gran exactitud utilizando las variables del rendimiento académico del alumno, variables del contexto familiar y variables del establecimiento.

Ahora, teniendo un prototipo del SPDE que cumple con los objetivos, es decir, predicción del fracaso escolar, la siguiente etapa de este trabajo es la creación de la plataforma de visualización o un sistema de información que apoye y facilite la información entregada por el modelo de predicción, y finalmente comenzar con la última etapa de la metodología de minería de datos CRISP-DM, que correspondería en la evaluación e implementación del modelo en el MINEDUC y las entidades relacionadas con la educación regular en Chile. También, debido a que las variables que explican la totalidad del poder de discriminación del modelo de predicción, se puede pensar en mejorar el modelo para lograr predecir la deserción escolar dentro de todo Chile y no solamente dentro del Gran Santiago. El trabajo presentado muestra que la implementación no tiene mayor complejidad y la información que les aportará a los tomadores de decisiones es un gran avance, en base a la forma actual de manejar los diferentes planes de acciones establecidos para combatir la deserción y el abandono escolar.
Finalmente, sería interesante realizar predicciones con mayor antelación de los posibles desertores, esto es por que como la deserción escolar es un proceso, entre antes se prediga la deserción, más barato y fácil les resultará intervenir para las entidades relacionadas a la educación. Lamentablemente para esto, es necesario mejorar la disponibilidad de información aumentando la frecuencia con que se entregan los datos públicos de la educación. Otro punto interesante también es entregar poder explicativo al modelo, como lo señalamos en el capitulo anterior, esto es sumamente difícil con las fuentes de información actuales, puesto que la información se encuentra muy agregada. Se hicieron los esfuerzos por crear indicadores que reflejen la información del alumno y su contexto de manera desagregada, pero concluimos que no es suficiente, es por esta razón, que también a futuro se plantea buscar alianzas con las entidades para poder entregar al modelo información más desagregada de los alumnos cosa de mejorar el poder explicativo del modelo.