\chapter{Definición de Línea Base}
\label{ch:lineabase}

En este capítulo se procederá a definir la línea base del problema, es decir, la forma en que se aborda la deserción escolar en Chile, los programas y políticas públicas enfocadas en la disminución de las tasas de deserción en el país y sus impactos.

\section{Situación Actual}
En esta sección se procede a caracterizar la situación actual con respecto a la deserción escolar en Chile, quiénes son los principales entes involucrados en su medición, cómo se enfrentan estas cifras y cuáles son los costos sociales para quienes deciden dejar el sistema de educación.

Para comenzar a comprender la situación actual de esta problemática en nuestro país, es necesario tener claridad con respecto a cómo se mide la deserción escolar en Chile. En el capítulo 2 se menciona que es el Ministerio de Educación quien, desde el año 2010, monitorea la deserción mediante dos tasas diferentes; la tasa de incidencia\footnote{Tasa de incidencia: Mide la deserción escolar evaluando la transición de un año a otro, por lo que se considera desertor al alumno que no retorna al sistema escolar luego de haber estado matriculado en el periodo académico anterior, sin que durante este periodo se haya graduado del sistema escolar} y la tasa de prevalencia\footnote{Tasa de prevalencia: Evalúa la deserción escolar desde una perspectiva estática ya que considera el estado presente de la persona analizada, y no la trayectoria educacional que ésta  ha  tenido. Permite conocer la proporción de jóvenes que no terminaron la educación escolar y que tampoco se encuentran matriculados en ningún establecimiento educacional en el periodo analizado}.

En base a estas mediciones, la tasa de incidencia de la deserción escolar presentada por el MINEDUC en el año 2011 era de un 3\%, lo que equivale a 91.968 alumnos que presentaban matrícula el año 2011 y no se gradúan ni se encuentran matriculados en el sistema regular de educación de niños y jóvenes el año 2012, pudiendo estar matriculados en la modalidad de educación de adultos. 

Para el mismo año, 2011, la tasa de prevalencia de la deserción registrada por el MINEDUC es de un 9,5\% para el rango de edad que considera entre los 15 y 19 años, quienes con esas edades, deberían encontrarse cursando enseñanza media. 

\section{Cómo se enfrenta la Deserción Escolar en Chile}
Según las cifras recién mencionadas, el Estado de Chile ha adoptado diferentes medidas para hacerle frente a esta situación, siendo la Junta Nacional de Auxilio Escolar y Becas, JUNAEB, junto al Ministerio de Educación y el Ministerio de Desarrollo Social, las instituciones que más intervienen en esta área de la educación. 

A continuación se procede a indagar en estos programas que se han creado con el fin de apoyar la retención escolar y así evitar que se genere el proceso de deserción. 

\subsection{Beca de Apoyo a la Retención Escolar}
La Beca de Apoyo a la Retención Escolar,BARE, consiste en la entrega de un aporte económico al estudiante, de libre disposición, por un monto anual\footnote{Para el año 20013, el monto anual de la beca fue de \$174.600.} que se distribuye en cuatro cuotas cuyo monto aumenta progresivamente hasta finalizar el año escolar.

Este beneficio está destinado a estudiantes de educación media, de establecimientos con altos niveles de vulnerabilidad socioeducativa, calificados de altamente vulnerables por condición de embarazo, maternidad, paternidad y aquellos beneficiarios del programa Chile Solidario. Se trata de un beneficio renovable, por lo que el estudiante que lo recibe podría mantenerlo por los cuatro años de enseñanza media mientras no se vea enfrentado a alguno de los criterios de supresión o eliminación.

El objetivo principal de la beca se encuentra orientado a favorecer la permanencia en el establecimiento educacional de los/as estudiantes de educación media con mayores niveles de vulnerabilidad, con el propósito de alcanzar los 12 años de escolaridad obligatoria, establecida en la Constitución Política de la República de Chile.

No existe postulación directa al beneficio, es decir, JUNAEB pre-asigna la beca a estudiantes de primero, segundo o tercero medio que cumplan con los criterios de condición de embarazo, maternidad, paternidad \footnote{Existe un sistema de registro de Embarazadas Junaeb y Sistema Chile Crece Contigo que buscan apoyar a las estudiantes en condición de embarazo adolescente que tienen riesgo de desertar del sistema eduacional. Chile Crece Contigo es un Sistema de Protección Integral a la Infancia que tiene como misión acompañar, proteger y apoyar integralmente, a todos los niños, niñas y sus familias, a través de acciones y servicios de carácter universal, así como focalizando apoyos especiales a aquellos que presentan alguna vulnerabilidad mayor: “a cada quien según sus necesidades”.\cite{chcct}} y/o aquellos/as beneficiarios/as del Programa Chile Solidario\footnote{El sistema de protección social Chile Solidario está dirigido a las familias y personas en situación de extrema pobreza y busca promover su incorporación a las redes sociales y su acceso a mejores condiciones de vida, para que superen la indigencia.} y de este grupo, a los/as que se ubiquen en primera, segunda, y hasta tercera prioridad según puntaje para la medición de la condición de Vulnerabilidad desarrollado por JUNAEB, denominado Índice de Vulnerabilidad Socioeducativa (IVSE). Siempre ajustándose a la disponibilidad presupuestaria y cupos que el decreto vigente o en trámite establece.\cite{bare}

Los requisitos para obtener esta beca son los siguientes: 
\begin{itemize}
\item Pertenecer al listado de estudiantes focalizados y pre-seleccionados por Junaeb.
\item Encontrarse matriculado/a y cursando de primero a tercer año de enseñanza media en un establecimiento de Educación Media Municipal o Particular Subvencionado al momento de la validación. Calidad que sólo será validada para quienes estén ingresados y vigentes en Sistema de Información de Estudiantes (SIGE) Mineduc.
\item Validación de antecedentes por parte del Encargado/a de la Beca Apoyo y Retención Escolar en el Establecimiento Educacional, referida a los antecedentes del Formulario del estudiante.
\item No posean alguna Beca de JUNAEB que sea incompatible con Beca BARE (Beca Indígena, Beca Presidente de la República, Beca Integración Territorial).
\end{itemize}

Para los estudiantes que ya obtuvieron la beca y desean mantenerla, deben cumplir los siguientes beneficios: 
\begin{itemize} 
\item Habiendo tenido la beca el año anterior, se encuentren matriculados de primer a cuarto año medio en un establecimiento de educación Media y registrados en el sistema informático Beca BARE en el año siguiente. (Se incluyen repitentes). Calidad que sólo será validada para quienes estén ingresados y vigentes en Sistema de Información de Estudiantes (SIGE) Mineduc.
\item No posean alguna Beca de JUNAEB que sea incompatible con Beca BARE (Beca Indígena, Beca Presidente de la República, Beca Integración Territorial).
\item Cumplan con el porcentaje de asistencia comprometida el año anterior o con el mínimo establecido por el programa del 85\%. Calidad que sólo será validada mediante Sistema de Información de Estudiantes (SIGE) Mineduc.
\item Mantengan la condición de vulnerabilidad socioeducativa que los hizo acreedor del beneficio. 
\end{itemize}

\subsection{Programa de Apoyo a la Retención Escolar}
El Programa de Retención Escolar, PARE, apoya el derecho a la educación que todo niño, niña y joven de Chile tiene garantizado gracias a la ley de obligatoriedad y gratuidad de la enseñanza media para todos los chilenos hasta los 21 años, promulgada en el año 2003.

JUNAEB, institución que busca contribuir y hacer efectiva igualdad de oportunidades, el desarrollo humano y la movilidad social, diseña y ejecuta programas de tipo promocional, preventivos y asistenciales que entregan apoyos integrales, favoreciendo así la igualdad de oportunidades ante la educación.

La creación del programa de retención escolar se alinea con este objetivo, contemplando como foco de intervención, el trabajo con todos los actores del sistema escolar a fin de promover la instalación de prácticas protectoras orientadas a la mantención y continuidad de los/as estudiantes en el sistema escolar que presentan riesgo socioeducativo, asociado ya sea por su condición de Embarazo, maternidad y paternidad, o bien, por otros factores de riesgo que incidan en la deserción escolar.

Consiste en apoyar pedagógica y psicológicamente a aquellos estudiantes vulnerables que se encuentran en riesgo de desertar del sistema escolar, lo que comprende intervenciones de caracter preventivo y promocional. Estas son realizadas por un equipo interdisciplinario (pedagogo, asistente social y psicólogo) a partir de intervenciones grupales e individuales en los ámbitos psicosociales, pedagógico, familiares y escolares; este último componente está orientado a trabajar con la comunidad educativa (estudiantes, directivos, profesores, apoderados) y redes sociales del territorio de cada estudiante.

Los recursos financieros contemplados para la ejecución en la modalidad de Escuelas Abiertas, tienen el carácter de fondo concursable, destinados al financiamiento de proyectos que se asignarán a entidades ejecutoras públicas y/o privadas sin fines de lucro.

La población objetivo del Programa la conforman los siguientes estudiantes en el siguiente orden de priorización:
\begin{enumerate}
\item Estudiantes de enseñanza básica (7 y 8) o Enseñanza media (1 a 4) en condición de embarazo, maternidad o paternidad, matriculados en establecimiento educacionales priorizados por vulnerabilidad socioeducativa que cuenten con la beca BARE.
\item Estudiantes de enseñanza básica (7 y 8) o Estudiantes de enseñanza media (1 a 4) que presenten alto riesgo socioeducativo, matriculados en establecimientos priorizados por vulnerabilidad socioeducativa que cuenten con la beca BARE.
\item Estudiantes de enseñanza básica (7 y 8) o Estudiantes de enseñanza media (1 a 4) que presenten alto riesgo socioeducativo, matriculados en establecimientos priorizados por vulnerabilidad socioeducativa que no cuenten con la beca BARE.
\end{enumerate}

Los beneficiarios del programa PARE son estudiantes, padres, madres y embarazadas adolescentes con riesgo socioeducativo pertenecientes a establecimientos de las regiones de Tarapacá, Antofagasta, Valparaíso, Metropolitana, O'Higgins, Maule, Bío Bío, Araucanía y Magallanes. \cite{pare}

\subsection{Programa Escolar 'Aquí, Presente'}
En mayo del 2015, la Subsecretaría de Educación junto a la Intendencia de la Región Metropolitana, lanzaron el programa escolar 'Aquí, Presente', iniciativa que busca mejorar la convivencia y apoyar la retención escolar en establecimientos educacionales de la Región Metropolitana.

La forma en que este programa se llevará a cabo es llevando duplas de profesionales del área psicosocial a los establecimientos públicos de la Región Metropolitana, con el objetivo de mejorar la convivencia y prevenir la deserción escolar. 

El programa 'Aquí, Presente' es financiado por el Fondo Nacional de Desarrollo Regional del Gobierno Metropolitano y es impulsado por la Intendencia Metropolitana. El programa, que se enmarca dentro del plan regional de Fortalecimiento de la Educación Pública, contempla la contratación de duplas psicosociales, que estarán conformadas por psicólogos, trabajadores sociales y pedagogos, quienes serán los encargados de identificar las señales de alerta de deserción en niños, niñas y adolescentes que presentan inasistencias reiteradas a los colegios con el fin de apoyarlos, además de mejorar la convivencia escolar y fortalecer el clima al interior del establecimiento.

\section{Conclusiones}
En esta sección fue posible conocer sobre las diversas formas en que el Estado, a través de instituciones como el Ministerio de Educación, se hace cargo de la deserción escolar, para lo que se han creado diferentes métodos para aabordar este fenómeno. 

También, es importante resaltar la labor de la Junta Nacional de Auxilio Escolar y Becas, JUNAEB, ya que los principales programas pro retención escolar provienen de esta institución. 

Finalmente, con respecto al programa 'Aquí, Presente', podemos ver que actualmente se siguie trabajando en la creación de políticas públicas que buscan disminuir las tasas de deserción. Si bien este es un programa muy reciente, será interesante conocer su efectividad a largo plazo. 