\documentclass[12pt,letterpaper,oneside]{book}

% Inicio de los paquetes
\setcounter{secnumdepth}{3}
\setcounter{tocdepth}{3}
\usepackage{geometry}
\usepackage[spanish]{babel}
\usepackage{import}
\usepackage{graphicx}
\usepackage{lipsum}
\graphicspath{ {Figuras/} }
\usepackage{array}
\usepackage{makeidx}
\usepackage{blindtext}
\usepackage{fancyhdr}
\usepackage[square, numbers, comma, sort&compress]{natbib}
\usepackage{color}
\usepackage{booktabs}
\usepackage{float}
\usepackage{hyperref}
\usepackage{amsmath}
\usepackage{amssymb}
\usepackage{enumitem}
\usepackage[utf8]{inputenc}
\usepackage[table,xcdraw]{xcolor}
\usepackage{multirow}
% Fin de los paquetes

% Inicio de funciones
\renewcommand{\headrulewidth}{0pt} % remove the header rule
\renewcommand{\baselinestretch}{1.5}
\newcommand{\seccion}[1]{\import{Secciones/}{#1.tex}}
\newcommand{\capitulo}[1]{\import{Capitulos/}{#1.tex}}
\newcommand{\apendice}[1]{\import{Apendices/}{#1.tex}}
\newcommand{\anexo}[1]{\import{Anexos/}{#1.tex}}
\newcommand{\cuadro}[1]{\import{Cuadros/}{#1.tex}}
\newcommand{\figura}[1]{\import{Figuras/}{#1.tex}}
\renewcommand{\thefootnote}{\arabic{footnote}}
% Fin de funciones

% Márgenes y otros
 \geometry{
 letterpaper,
 left=35mm,
 right=15mm,
 top=20mm,
 bottom=20mm,
 }
\fancyhf{}
\fancyfoot[R]{\thepage}
\pagestyle{fancy}
\fancypagestyle{plain}{%
    \renewcommand{\headrulewidth}{0pt}%
    \fancyhf{}%
    \fancyfoot[R]{\thepage}%
}

% Fin de Márgenes y otros

\makeindex
\begin{document}

\frontmatter
\begin{titlepage}
\includegraphics[width=0.4\textwidth]{UAI}
    \begin{center}
        \vspace*{5cm}
        
        \textbf{“Desarrollo de un sistema prototipo para la detección temprana de la deserción escolar en escuelas públicas chilenas”}
        
        \vspace{4cm}

        
        \textbf{Camila Paz Escobar Calderón}\\
        \textbf{Felipe Elías Lolas Isuani}\\
        \begin{flushright}
        Profesor Tutor: Karol Suchan
        \end{flushright}
        \vfill
        
        \textbf{PROPUESTA DE MEMORIA PARA OPTAR AL TÍTULO DE INGENIERO CIVIL INDUSTRIAL}

        
        \vspace{0.8cm}
        
 
        \textbf{2015}
        
    \end{center}
\end{titlepage}
\clearpage
\thispagestyle{empty}
%\addcontentsline{toc}{chapter}{Resumen Ejecutivo}
    \begin{center}
        \vspace*{1cm}
        
        \textbf{Resumen Ejecutivo}
        
        \vspace{0.5cm}
        
    \end{center}


En el presente documento se aborda la deserción escolar dentro de la educación pública de Chile. El objetivo es la creación de un sistema de alerta temprana que apoye la toma de desiciones que se toman en el Ministerio de Educación, en cuanto a políticas públicas dirigidas a la educación pública del país. 

Para lograr lo propuesto se trabajará en conjunto al Centro de Investigación Avanzada en Educación de la Universidad de Chile. 

Se trabajarán diversas fuentes que incluyen información tanto de los establecimientos públicos de Santiago, sus alumnos y las familias de estos. Para esto se utilizarán herramientas de Minería de Datos y \textit{Big Data}.
\clearpage
\thispagestyle{empty}
\tableofcontents
\thispagestyle{empty}
\clearpage

\setcounter{page}{1}
\listoffigures
\addcontentsline{toc}{chapter}{Índice de figuras}
\clearpage
\listoftables
\addcontentsline{toc}{chapter}{Índice de cuadros}
\clearpage

\mainmatter
\capitulo{1Introduccion1}
\capitulo{2DescripcionDelProblema}
\capitulo{3MarcoTeorico}
\capitulo{4Metodologia}
\capitulo{5LineaBase}
\capitulo{6SolucionPropuesta}
\capitulo{7AnalisisDeResultados}
\capitulo{8ConclusionesRecomendaciones}
\capitulo{9TrabajoFuturo}
\lipsum[1]
\chapter*{Anexo}
\addcontentsline{toc}{chapter}{Anexo}
\renewcommand{\thesection}{\Alph{section}}
\anexo{AnexoA}
\anexo{AnexoB}
\anexo{AnexoC}
\appendix
\addcontentsline{toc}{chapter}{Apéndice}
\chapter*{Apéndice}
\apendice{ApendiceA}
\backmatter
\bibliographystyle{unsrtnat}
\bibliography{bibliografia}
\end{document}
