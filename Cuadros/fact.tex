\begin{table}[h]
\centering
\scalebox{0.7}{
\resizebox{\textwidth}{!}{%
\begin{tabular}{ll}
\hline
Predictores de Deserción Escolar & Efecto Observado \\ \hline
\textit{Factores estudiantiles} &  \\
\  \**Factores psicológicos y de comportamiento &  \\
\    -Habilidad/logro académico & -Si es mayor, el riesgo de deserción escolar es menor \\
\    -Grado de retención/repetición & -Si es el caso, mayor riesgo de deserción escolar \\
\    -Aspiraciones educacionales y ocupacionales & -Si es mayor, menor riesgo de deserción escolar \\
\    -Compromiso & \begin{tabular}[c]{@{}l@{}}-A mayor ausentismo y/o problemas de disciplina,\\  mayor riesgo de deserción escolar\end{tabular} \\
\    -Carga académica de estudiantes de secundaria & \begin{tabular}[c]{@{}l@{}}-Si es intensiva, inadecuada, estresante e inestable,\\  mayor riesgo de deserción escolar\end{tabular} \\
\    -Matrimonio y embarazo adolescente & -Resultados mixtos \\
\  \**Factores demográficos &  \\
\    -Género & -Resultados mixtos \\
\    -Raza/etnia & -Resultados mixtos \\
\    -Estado de inmigración & -Resultados mixtos \\
\    -Idioma & -Si es nativo, menor riesgo de deserción escolar \\
\    -Discapacidad & -Si es el caso, mayor riesgo de deserción escolar \\
\textit{Factores familiares} &  \\
\  \**Características estructurales &  \\
\    -Status socio-económico & \begin{tabular}[c]{@{}l@{}}-Si es menor, entonces hay mayor riesgo de deserción\\  escolar\end{tabular} \\
\    -Estructura familiar & -Sin efecto independiente \\
\  \**Procesos subyacentes &  \\
\    -Capital social & -Si es mayor, menos riesgo de deserción escolar \\
\    -Capital humano/cultural & \begin{tabular}[c]{@{}l@{}}-Si es mayor, menor riesgo de deserción, pero sin\\  efecto independiente\end{tabular} \\
\    -Capital financiero & -Sin efecto independiente \\
\textit{Factores escolares} &  \\
\    -Tipo de escuela & \begin{tabular}[c]{@{}l@{}}-Si es pública y sin selectividad, alto riesgo de\\  deserción de sus estudiantes\end{tabular} \\
\    -Recursos de la escuela & -Si estos están balanceados, menor riesgo de deserción \\
\    -Características estructurales de la escuela & -Sin efecto independiente \\
\    -Prácticas y políticas de la escuela & \begin{tabular}[c]{@{}l@{}}-Si es pequeña, menor riesgo de deserción, pero\\  quizás sin efecto independiente\end{tabular} \\
\   \**Clima social y académico & -Si es estimulante, menos riesgo de deserción escolar \\
\   \**Calidad de profesores y de enseñanza & -Si es mayor, el riesgo de deserción es menor \\
\   \**Capital social de la escuela & -Si es mejor, menor riesgo de deserción escolar \\
\textit{Factores de la comunidad} &  \\
\    -Características de los vecinos & -Si es perjudicial, mayor riesgo de deserción  escolar \\
\    -Alto rendimiento vs. deserción de los amigos & \begin{tabular}[c]{@{}l@{}}-Menor y mayor riesgo de deserción escolar,\\  respectivamente\end{tabular} \\
\    -Oportunidades de empleo &  \\
\      -Escasez de empleo y bajos salarios & \begin{tabular}[c]{@{}l@{}}-Si el empleo es escaso, menor riesgo de deserción\\  escolar\end{tabular} \\
\      -Largas horas de trabajo & \begin{tabular}[c]{@{}l@{}}-Si las horas de trabajo superan las 20 horas, el riesgo\\  de deserción escolar aumenta\end{tabular} \\
\    -Discriminación social/injusticia & -Si es el caso, mayor riesgo de deserción \\ \hline
\end{tabular}
}
}
\caption{Factores predictivos de deserción escolar temprana según literatura disponible\cite{EDMSurv2}}
\label{tab:fact}
\end{table}
