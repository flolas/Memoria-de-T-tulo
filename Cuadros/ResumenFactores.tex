% Please add the following required packages to your document preamble:
% \usepackage{booktabs}
% \usepackage{graphicx}
% \usepackage[table,xcdraw]{xcolor}
% If you use beamer only pass "xcolor=table" option, i.e. \documentclass[xcolor=table]{beamer}
\begin{table}[H]
\centering
\resizebox{\textwidth}{!}{%
\begin{tabular}{@{}|l|l|@{}}
\toprule
\rowcolor[HTML]{C0C0C0} 
\multicolumn{1}{|c|}{\cellcolor[HTML]{C0C0C0}\textbf{Documento}} & \multicolumn{1}{c|}{\cellcolor[HTML]{C0C0C0}\textbf{Factores relevantes}} \\ \midrule
Comprendiendo el fenoomeno de la desercion escolar en Chile. & \begin{tabular}[c]{@{}l@{}}Este documento plantea la existencia de 5 grupos de variables que afectan en la deserción \\ escolar: sociales (tener amigos dentro de la escuela y la relación con sus compañeros de \\ curso y con las autoridades del colegio), familiares (escolaridad de la madre, consumo de \\ alcohol de la madre, violencia física grave de la madre con el joven y responsabilidad del \\ joven con sus familiares),  de salud (nivel de auto percepción de su salud, especialmente \\ psicológica), conductuales (consumo de drogas y alcohol, embarazo temprano) y \\ educacionales (repitencia, inasistencia, malas calificaciones, baja importancia asignada al estudio).\end{tabular} \\ \midrule
\begin{tabular}[c]{@{}l@{}}Adolescentes y jóvenes que abandonan sus estudios\\ antes de finalizar la enseñanza media: principales tendencias.\end{tabular} & \begin{tabular}[c]{@{}l@{}}Los resultados de este estudio muestran que, para los niños y jóvenes de entre 14 y 17años, los\\  factores que influyen en la deserción escolar son: el ser padre o madre, el atraso escolar, un \\ alto número de personas en el hogar del alumno, bajos ingresos per capita, vivir en zonas \\ rurales, vivir en zonas donde hay baja oferta educacional (número de establecimientos) y la no \\ asistencia.\end{tabular} \\ \midrule
Dinámica de la deserción escolar en Chile. & \begin{tabular}[c]{@{}l@{}}Dentro de las variables que se analizaron, aquellas que se encontró que afectaban la deserción \\ escolar se encuentra el género del alumno, la maternidad/paternidad, la estructura familiar del \\ alumno, el ingreso del hogar,,la escolaridad del jefe de hogar y la tasa de cobertura de \\ educación media en la comuna del alumno. Además, se concluye que el riesgo de deserción \\ se concentra principalmente en el ciclo secundario.\end{tabular} \\ \midrule
\begin{tabular}[c]{@{}l@{}}Procesos de deserción en la enseñanza media. Factores\\ expulsores y protectores.\end{tabular} & \begin{tabular}[c]{@{}l@{}}Se presenta a los pares como un factor que incentiva la deserción o como un factor de apoyo \\ que justifica permanecer en el establecimiento. El papel de la familia para los y las jóvenes \\ desertores,  en el caso de estar ausente actúa como un factor de expulsión del sistema escolar. \\ En el caso contrario, para los no desertores, la presencia de la familia funciona como un \\ factor de protección. La deserción aparece vinculada también con la experiencia del joven \\ en el establecimiento y también con la “oferta de liceos” que existe en su lugar de residencia. \\ En un mismo establecimiento coexisten medidas expulsoras, como bajo rendimiento o \\ problemas de conducta, y protectoras, como las buenas relaciones que establecen profesores \\ con alumnos. La sobre la edad, el cumplir 18 años y estar en segundo medio es un factor que \\ casi mecánicamente implica un caso de deserción.\end{tabular} \\ \midrule
\begin{tabular}[c]{@{}l@{}}Características de la población juvenil desertora del sistema\\ escolar chileno.\end{tabular} & \begin{tabular}[c]{@{}l@{}}Plantea la existencia de un conjunto de factores observables previos al proceso de deserción, \\ como alta frecuencia de cambios de colegio, resistencia, problemas conductuales, bajo éxito \\ académico, baja asistencia a clases y mayores atrasos.\end{tabular} \\ \midrule
\begin{tabular}[c]{@{}l@{}}Factores familiares asociados a la deserción escolar\\ en Chile.\end{tabular} & \begin{tabular}[c]{@{}l@{}}Este documento reconoce la existencia de dos grandes grupos de factores que generan \\ deserción, siendo el foco aquellos factores extraescolares como la situación socioeconómica, \\ contexto familiar de niños, niñas y jóvenes, la pobreza y la marginalidad, la búsqueda de \\ trabajo, el embarazo adolescente, la disfuncionalidad familiar, el consumo de drogas y las \\ bajas expectativas de la familia con respecto a la educación, entre otros. Con respecto a las \\ variables familiares explicativas, se confirma la existencia del vínculo entre el nivel de \\ enseñanza alcanzado del apoderado y el abandono escolar, existencia de problemas \\ económicos lo que no genera necesariamente la inclusión de los niños/as al mundo laboral, \\ la estructura de la familia también afectaría al proceso de deserción ya que la mayoría de \\ quienes desertan se encuentran insertos en familias donde uno de los progenitores está \\ ausente, las familias de los desertores son muchas veces numerosas (lo que no afecta el \\ rendimiento de los niños sino que se vincula con la existencia de problemas al interior de \\ las familias).\end{tabular} \\ \midrule
\begin{tabular}[c]{@{}l@{}}Deserción escolar en Chile: un estudio de caso\\ con factores intraescolares\end{tabular} & \begin{tabular}[c]{@{}l@{}}Plantea que los factores que originan este problema se dividen en dos grandes grupos: \\ intraescolares y extraescolares. Se enfoca en aquellos factores intraescolares, como \\ problemas conductuales, bajo rendimiento académico, autoritarismo docente y \\ adultocentrismo. Así mismo, se plantea que las repitencias, expulsiones y la sobreedad \\ del alumnado actúan como antesala de la deserción definitiva, y son más frecuentes en \\ las instituciones educativas que atienden a sectores socioeconómicos de bajos ingresos.\end{tabular} \\ \midrule
\begin{tabular}[c]{@{}l@{}}El ambiente escolar incide en los resultados PISA\\ 2009: Resultados de un estudio de diseño mixto.\end{tabular} & \begin{tabular}[c]{@{}l@{}}Se crea un Índice de Ambiente Escolar (IAE) que incluye las dimensiones de \\ valoración positiva del establecimiento, apoyo de profesores, autonomía, \\ participación y expectativas positivas de los estudiantes y sus familias, y fue \\ construido considerando su peso explicativo en el rendimiento educativo.\end{tabular} \\ \bottomrule
\end{tabular}
}
\caption{Resumen de los factores que influyen en la deserción escolar encontrados en la literatura.}
\label{tab:resumen}
\end{table}