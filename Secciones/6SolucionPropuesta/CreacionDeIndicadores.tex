\section{Creación de Indicadores}
En base a la información contenida en los \textit{datasets} disponibles, se procede a generar una lista de indicadores. 

La finalidad de crear estos indicadores radica en poder resumir información para luego poder utilizarla en el modelo de predicción. Esto es útil ya que, al ser tanta la información disponible, se pueden optimizar los tiempos de procesamiento, disminuyendo el número de variables de entrada. 

Los indicadores que se crearon abarcan información personal del alumno como la sobre edad, que epresenta la diferencia entre la edad de un alumno y la edad esperada para el nivel que se encuentra cursando, si es repitente, su ranking de rendimiento, entre otros\footnote{Ver Anexo.}. 

Por otro lado, se aborda lo que se relaciona con el contexto intraescolar, como la cantidad promedio de alumnos por curso, y aquello relacionado con los docentes del establecimiento, como el promedio de horas lectivas\footnote{Ver Anexo.}.

Finalmente, estos indicadores capturan también información sobre el ambiente extraescolar y de la familia de cada alumno\footnote{Ver Anexo.}. 
