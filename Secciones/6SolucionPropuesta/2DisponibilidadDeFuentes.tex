\section{Disponibilidad de Información}
Las diferentes entidades del sistema educativo chileno y grupos de investigadores dedicados a la educación preparan y desarrollan diferentes fuentes de información que serán cruciales para el desarrollo de este proyecto.

Gracias a las fuentes de información mencionadas recientemente, y a la Ley de transparencia descrita en el capítulo 1, es posible tener acceso a la información recaudada.

A continuación se muestra en el Cuadro la descripción de las fuentes de información y en la Figurase muestra la disponibilidad de estos.

Al ver la Figura podemos destacar y mencionar que asegurando la calidad de estas fuentes de información y la creación de diferentes indicadores relevantes que indiquen deserción escolar se puede desarrollar una herramientas potente, pues se tiene información de el rendimiento, asistencia, características socio-económicas familiares de estudiantes, junto con caracterizaciones de vulnerabilidad, delitos y ambiente de las diferentes comunas, e inclusive isócronas\footnote{Una isócrona es un área geométrica en un mapa que muestra individuos con una misma característica, como por ejemplo misma manzana} de la manzana del colegio y/o estudiante.


\subsection{Bases de Datos}
A continuación se procede a describir las bases de datos generadas por las diferentes fuentes de información. 

En la Tabla~\ref{tab:datasets} se presentan las diferentes bases de datos, su nombre, su origen, para qué años está disponible y una breve descripción de su contenido.

\begin{table}[H]
\centering
\resizebox{\textwidth}{!}{%
\begin{tabular}{@{}|c|c|c|c|@{}}
\toprule
\textbf{Nombre} & \textbf{Fuente} & \textbf{Años Disponibles} & \textbf{Descripción} \\ \midrule
\begin{tabular}[c]{@{}c@{}}Asignación Exelencia \\ Pedagógica (AEP)\end{tabular} & CPEIP & 2002 - 2013 & \begin{tabular}[c]{@{}c@{}}Esta base de datos presenta, para cada docente, sus puntajes obtenidos en las \\ diferentes pruebas que se les aplica, su puntaje final y su estadofinal \\ de acreditación.\end{tabular} \\ \midrule
Asistencia Mensual & SIGE & 2011 - 2015 & \begin{tabular}[c]{@{}c@{}}Para cada mes de cada año, esta base de datos da a conocer los días asistidos, los días \\ de clases y el porcentaje de asistencia para cada niño (identificado con un MRUN) \\ de cada establecimiento.\end{tabular} \\ \midrule
Censo Docente & \begin{tabular}[c]{@{}c@{}}SIDOC (hasta 2010)\\ SIGE (desde 2011)\end{tabular} & 2003 - 2014 & \begin{tabular}[c]{@{}c@{}}Para cada docente (identificado con una CLAVE) se da a conocer información como \\ género,año de nacimiento,  título y si tiene 1 o más, menciones (artes plásticas, ciencias\\  naturales, lenguaje, matemática, educación física, religión, etc),  en qué tipo de institución \\ estudió, año de titulación, su función principal en el establecimiento que se encuentra\\  trabajando, el tipo de contrato, horas de trabajo, años de servicio en el establecimiento,\\  grados en los que hace clases, entre otros.\end{tabular} \\ \midrule
Dotación Docente & \begin{tabular}[c]{@{}c@{}}SIDOC (hasta 2010)\\ SIGE (desde 2011)\end{tabular} & 2003 - 2014 & \begin{tabular}[c]{@{}c@{}}En esta base de datos, se resume la dotación docente de cada establecimiento, es decir, la \\ totalidad de docentes como función principal y sus horas de contrato, total de docentes en \\ la planta directiva y sus horas de contrato, total de docentes como directores y sus horas \\ de contrato, total de inspectores y sus horas de contrato, total de orientadores y sus horas \\ de contrato, entre otros.\end{tabular} \\ \midrule
Encuesta SIMCE & Agencia de Calidad & 1998 - 2014 & \begin{tabular}[c]{@{}c@{}}La encuesta SIMCE se aplica a todos los alumnos que rinden la prueba, a sus apoderados \\ y docentes. En ellos se busca capturar información de cada uno de estos actores y sus \\ percepciones frente al establecimiento y su ambiente.\end{tabular} \\ \midrule
\begin{tabular}[c]{@{}c@{}}Otros Indicadores de \\ Calidad de \\ Educación (OIC)\end{tabular} & Agencia de Calidad & 2010 - 2014 & \begin{tabular}[c]{@{}c@{}}Se presentan los resultados, para cada establecimiento, de tres de los ocho OIC: autoestima \\ académica y motivación escolar, clima de convivencia escolar, y participación y formación \\ ciudadana.\end{tabular} \\ \midrule
\begin{tabular}[c]{@{}c@{}}Información Territorial\\  por Isocrona\end{tabular} & CIT & 2013 & \begin{tabular}[c]{@{}c@{}}En estas bases de datos se presenta información geolocalizada de grupo socioeconómico, \\ niveles de delincuencia, cantidades de áreas verdes, cultura, usos de suelo, entre otros. \\ Cuando se habla de isocrona se refiere a la construcción de polígonos que representan \\ 5, 10 o 15 minutos a pie, en automóvil o en transporte público.\end{tabular} \\ \midrule
IVE-Sinae & JUNAEB & 2010 - 2015 & El IVE-SINAE se presenta para cada ciclo educacional (básica y media) de cada comuna del país. \\ \midrule
\begin{tabular}[c]{@{}c@{}}Matrícula por \\ Establecimiento\end{tabular} & \begin{tabular}[c]{@{}c@{}}RECH (hasta 2008)\\ SIGE (desde 2009)\end{tabular} & 2004 - 2014 & \begin{tabular}[c]{@{}c@{}}En esta base de datos se presenta, para cada establecimiento, el lugar donde se encuentra, \\ el total de matrículas por género, por grado y por nivel de enseñanza (básica y media).\end{tabular} \\ \midrule
\begin{tabular}[c]{@{}c@{}}Matrícula por \\ Estudiante\end{tabular} & \begin{tabular}[c]{@{}c@{}}RECH (hasta 2008)\\ SIGE (desde 2009)\end{tabular} & 2004 - 2014 & \begin{tabular}[c]{@{}c@{}}Esta base de datos da a conocer a los estudiantes (caracterizados por un MRUN),  el establecimiento \\ al que asiste y el grado que cursa, junto a su fecha de nacimiento y el lugar de residencia.\end{tabular} \\ \midrule
\begin{tabular}[c]{@{}c@{}}Matrícula y \\ Geolocalización \\ RM\end{tabular} & CIT & 2013 - 2014 & \begin{tabular}[c]{@{}c@{}}El CIT preparó esta base de datos que contiene la geolocalización de cada alumno matriculado\\  en un establecimiento.  Si bien en Matrícula por Estudiante se encuentra el lugar de residencia \\ del alumno, en este dataset se da a conocer el posicionamiento del alumno en el sistema \\ de coordenadas geográficas (latitud y longitud).\end{tabular} \\ \midrule
\begin{tabular}[c]{@{}c@{}}Rendimiento por \\ Estudiante\end{tabular} & \begin{tabular}[c]{@{}c@{}}RECH (hasta 2008)\\ SIGE (desde 2009)\end{tabular} & 2002 - 2014 & \begin{tabular}[c]{@{}c@{}}Esta base de datos contiene, para cada alumno, el establecimiento al que asiste, el grado\\  que cursa, su género, su comuna de residencia, su promedio general anual, su porcentaje \\ de asistencia y su situación de promoción al final del año escolar (promovido, reprobado, \\ retirado, trasladado).\end{tabular} \\ \midrule
Resultados PSU & MINEDUC & 2006 - 2012 & \begin{tabular}[c]{@{}c@{}}Contiene indicadores para promedio PSU de Lenguaje, Matemática y NEM \\ por establecimiento.\end{tabular} \\ \midrule
Resultados SIMCE & Agencia de Calidad & 1998 - 2014 & \begin{tabular}[c]{@{}c@{}}Se dan a conocer los resultados tanto de las pruebas SIMCE con sus puntajes por \\ establecimiento, como de los cuestionarios SIMCE de estudiantes, docentes\\  y apoderados con sus percepciones.\end{tabular} \\ \midrule
\begin{tabular}[c]{@{}c@{}}Sistema Nacional \\ Evaluación del \\ Desempeño \\ (SNED)\end{tabular} & UNAM & 1996 - 2015 & \begin{tabular}[c]{@{}c@{}}En esta base de datos se presenta, para cada establecimiento, sus puntajes de efectividad, \\ superación, iniciativa, mejoramiento, integración e igualdad (que van entre 0 y 100), la \\ clasificación de cada uno de ellos en clústers y índice SNED. Además, indica si fueron \\ premiados con subvención (100\% o 60\%).\end{tabular} \\ \bottomrule
\end{tabular}
}
\caption{Bases de datos, origen, disponibilidad y breve descripción.}
\label{tab:datasets}
\end{table}
