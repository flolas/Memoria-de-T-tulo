\section{Entendimiento del Negocio y los Datos}
\subsection{Fuentes de Información}
En este apartado se presentan las fuentes de información que proveerán datos que respaldarán el desarrollo del modelo de predicción. 

Previo a esto, es necesario tener claridad de qué información contienen, cómo se obtienen estos datos y sus disponibilidades, para luego definir qué se utilizará y que no. Además, de cómo se pre procesaran toda esta información, con el fin de evitar errores y datos anómalos. 

\subsubsection{Sistema de Información General de Estudiantes}
El Sistema de Información General de Estudiantes, SIGE, es una plataforma web creada por el Ministerio de Educación de Chile que tiene como objetivo integrar en un solo lugar toda la información referente a sostenedores, establecimientos educacionales, docentes y alumnos. 
En este sistema participan todo tipo de establecimientos; municipales, particulares subvencionados y particulares pagados con el fin de hacer llegar la información constantemente. 
Esta información integra procesos asociados, tales como:
\begin{itemize}
\item Matrícula inicial
\item Declaración de Asistencia
\item Actas de rendimiento 
\item Idoneidad docente
\item Rendimiento 
\item Ingreso de Asistencia
\end{itemize}

\subsubsection{Sistema de Idoneidad Docente}
El Sistema de Idoneidad Docente, SIDOC, corresponde a la sistematización de la información relativa al cuerpo docente en ejercicio profesional en cada establecimiento educacional del país. Es un registro censal y anual del Ministerio de Educación de Chile, que consiste en una plataforma implementada a partir del año 2003 y que incluye información de los años 2003 a 2010, ya que del año 2011 en adelante es el SIGE quien proporciona estos datos.

Este sistema contiene información sobre género, edad, título y mención, 
La información que contiene este sistema es: género, edad, título y mención, función que realiza en el establecimiento, nivel y asignatura que realiza cada docente aula, horas y tipo de contrato. A su vez, contiene información respecto a los asistentes de la educación, es decir, sobre los auxiliares, administrativos, y otros profesionales no docentes del establecimiento. 

\subsubsection{Registro de Estudiantes de Chile}
El Registro de Estudiantes de Chile, RECH, es un sistema de información extranet que captura, almacena y administra datos personales y académicos de cada uno de los estudiantes de enseñanza básica y media del país. 

Surge entre los años 2001 y 2002 con el fin de automatizar, normalizar, organizar y mantener una gran base de datos del sistema escolar para generar información de calidad para posteriores estudios, investigaciones y servicios propios del MINEDUC y de otros actores.

Este sistema contiene información relativa a los estudiantes desde el año 2002, y estos datos comprenden principalmente: 
\begin{itemize}
\item Identificación del estudiante (RUN)
\item Establecimiento educacional en que se educa
\item Plan de estudio que recibe
\item Equipo docente responsable
\item Desempeño académico alcanzado
\end{itemize}

Existen dos procesos que alimentan con información a este sistema, que exigen la participación de todos los establecimientos educacionales del país oficialmente reconocidos y de todos los sectores (públicos y privados). Estos son:
\begin{enumerate}
\item Matrícula inicial: consiste en la captura de información al inicio de cada año escolar, de todos y cada uno de los grados y establecimientos del país. Esta información se registra al 30 de abril, contiene datos personales de cada alumno (nombre, RUN, apellidos, fecha de nacimiento y comuna de residencia) y datos sobre el establecimiento educacional asociado al alumno como su modalidad de enseñanza y curso.
\item Rendimiento escolar: se trata de la captura de información al final de cada año escolar, de todos y cada uno de los grados y establecimientos del país.Se recogen los datos sobre el plan de estudio o malla curricular, el cuerpo docente por asignatura y el rendimiento académico del alumno ( lo que incluye nota final de cada asignatura, promedio final, \% de asistencia, observaciones y situación final de promoción que puede ser retirado, promovido o reprobado).
\end{enumerate}

\subsubsection{Centro de Perfeccionamiento, Experimentación e Investigaciones Pedagógicas}
El Centro de Perfeccionamiento, Experimentación e Investigaciones Pedagógicas, CPEIP, es el organismo del Ministerio de Educación encargado de diseñar, implementar y evaluar todo lo referente a los docentes, equipos directivos y asistentes de la educación.

El CPEIP es responsable de llevar un Registro Público Nacional de Perfeccionamiento de todos los cursos que se ofrecen a los profesores dependientes del sector municipal, con el fin que puedan impetrar el reconocimiento y pago de la Asignación de Perfeccionamiento.
 
En este registro, se inscriben cursos realizados por el CPEIP, por las Instituciones de Educación Superior autónomas y por las Instituciones Públicas o Privadas debidamente acreditadas por el CPEIP.
 
Además tiene la obligación de difundir los cursos inscritos de manera que los profesores tengan información para escoger adecuadamente el curso que les interesa profesionalmente y los municipios puedan verificar la real existencia de los cursos nominados en los certificados que le entregan los profesionales de la educación de su dependencia.

\subsubsection{Agencia de Calidad}
La Agencia de Calidad, mencionada en el apartado 1.1.3 del Capítulo 1 del presente documento, es el encargado del SIMCE y de toda la información que de esta prueba se obtiene. 

Dentro de sus registros, se genera información a tres niveles diferentes: 
\begin{itemize}
\item Estudiantes: dan origen a información de dos índoles; por un lado se obtienen los resultados de las pruebas SIMCE, con sus puntajes, y por otro lado se obtienen los resultados de cuestionarios aplicados a los mismos estudiantes evaluados, en los que se indaga sobre aspectos generales de su establecimiento escolar, así como sobre sus procesos y estrategias de aprendizaje dentro y fuera del aula, sus intereses y motivaciones, entre otros temas.
\item Docentes: resultados de cuestionarios respondidos por los profesores de cada curso evaluado y contienen preguntas relativas a su formación profesional y a los contenidos que enseñó durante el año escolar, entre otros aspectos.
\item Apoderados: resultados de cuestionarios respondidos por los apoderados de cada estudiante evaluado, donde se indaga sobre el nivel educacional de los padres, ingreso del hogar y nivel de satisfacción con el establecimiento, entre otros.
\end{itemize}

\subsubsection{Junta Nacional de Auxilio Escolar y Becas}
La Junta Nacional de Auxilio Escolar y Becas, JUNAEB, también actúa como una fuente de información, aportando con el indicador IVE-SINAE. Este permite medir la condición de vulnerabilidad de todos los alumnos, y se construye con insumos de diferentes fuentes de información de cada estudiante y que llegan a JUNAEB mediante Convenios interinstitucionales, como encuestas de vulnerabilidad JUNAEB, sistema de afiliación de salud, pertenecer a algun programa de la red SENAME, entre otros. 

\subsubsection{Unidad de Apoyo Municipal}
La Unidad de Apoyo Municipal, UNAM, fue creado en el año 2010 y es el representante del Ministerio de Educación frente a las relaciones con las municipalidades, en los aspectos de gestión, prestando asesoría a quienes lo requieran con el fin de generar alternativas eficaces de mejora en la gestión administrativa, financiera y técnica del área de educación de cada municipio.

En la UNAM existen dos áreas de trabajos, una de ellas está dedicada a la administración de programas específicos para municipios, como el Fondo de Apoyo a la Gestión, FAGEM, el Bono al Retiro Docente, el Fondo para Docentes Jubilados, Fondos Municipales de Gestión, el Fondo de Reconversión, el Sistema Nacional de Evaluación del Desempeño, SNED, y en general las transferencias de recursos.  

Con respecto a este último, el Sistema Nacional de Evaluación del Desempeño, son los encargados de proveer la información referente al desempeño de los profesionales de la educación. 

\subsubsection{Centro de Inteligencia Territorial}
El Centro de Inteligencia Territorial, CIT, de la Universidad Adolfo Ibáñez es el encargado de generar bases de datos territorializadas de alta calidad mediante la incorporación de nuevas tecnologías y gracias a la colaboración de diversas entidades. 

Dentro de la información que reúnen, almacenan y gestionan, está el grupo socioeconómico, niveles de delincuencia, cantidades de áreas verdes, cultura, usos de suelo, entre otros. Toda esta información se encuentra geolocalizada, por lo que permite conocer estas características en diferentes sectores del país.

\subsection{Disponibilidad de Información}
Las diferentes entidades del sistema educativo chileno y grupos de investigadores dedicados a la educación preparan y desarrollan diferentes fuentes de información que serán cruciales para el desarrollo de este proyecto.

Gracias a las fuentes de información mencionadas recientemente, y a la Ley de transparencia descrita en el capítulo 1, es posible tener acceso a la información recaudada.

\subsubsection{Bases de Datos}
A continuación se procede a describir las bases de datos generadas por las diferentes fuentes de información. 

En la Tabla~\ref{tab:datasets2} se presentan las diferentes bases de datos, su nombre, su origen, para qué años está disponible y una breve descripción de su contenido.

Al ver disponibilidad de bases de datos, podemos destacar y mencionar que, asegurando la calidad de estas fuentes de información y la creación de diferentes indicadores relevantes que indiquen deserción escolar, es posible desarrollar una herramientas potente, pues se tiene información de el rendimiento, asistencia, características socio-económicas familiares de estudiantes, junto con caracterizaciones de vulnerabilidad, delitos y ambiente de las diferentes comunas, e inclusive isócronas\footnote{Una isócrona es un área geométrica en un mapa que muestra individuos con una misma característica, como por ejemplo misma manzana} de la manzana del colegio y/o estudiante.

\begin{table}[H]
\centering
\resizebox{\textwidth}{!}{%
\begin{tabular}{@{}|c|c|c|c|@{}}
\toprule
\textbf{Nombre} & \textbf{Fuente} & \textbf{Años Disponibles} & \textbf{Descripción} \\ \midrule
\begin{tabular}[c]{@{}c@{}}Asignación Exelencia \\ Pedagógica (AEP)\end{tabular} & CPEIP & 2002 - 2013 & \begin{tabular}[c]{@{}c@{}}Esta base de datos presenta, para cada docente, sus puntajes obtenidos en las \\ diferentes pruebas que se les aplica, su puntaje final y su estadofinal \\ de acreditación.\end{tabular} \\ \midrule
Asistencia Mensual & SIGE & 2011 - 2015 & \begin{tabular}[c]{@{}c@{}}Para cada mes de cada año, esta base de datos da a conocer los días asistidos, los días \\ de clases y el porcentaje de asistencia para cada niño (identificado con un MRUN) \\ de cada establecimiento.\end{tabular} \\ \midrule
Censo Docente & \begin{tabular}[c]{@{}c@{}}SIDOC (hasta 2010)\\ SIGE (desde 2011)\end{tabular} & 2003 - 2014 & \begin{tabular}[c]{@{}c@{}}Para cada docente (identificado con una CLAVE) se da a conocer información como \\ género,año de nacimiento,  título y si tiene 1 o más, menciones (artes plásticas, ciencias\\  naturales, lenguaje, matemática, educación física, religión, etc),  en qué tipo de institución \\ estudió, año de titulación, su función principal en el establecimiento que se encuentra\\  trabajando, el tipo de contrato, horas de trabajo, años de servicio en el establecimiento,\\  grados en los que hace clases, entre otros.\end{tabular} \\ \midrule
Dotación Docente & \begin{tabular}[c]{@{}c@{}}SIDOC (hasta 2010)\\ SIGE (desde 2011)\end{tabular} & 2003 - 2014 & \begin{tabular}[c]{@{}c@{}}En esta base de datos, se resume la dotación docente de cada establecimiento, es decir, la \\ totalidad de docentes como función principal y sus horas de contrato, total de docentes en \\ la planta directiva y sus horas de contrato, total de docentes como directores y sus horas \\ de contrato, total de inspectores y sus horas de contrato, total de orientadores y sus horas \\ de contrato, entre otros.\end{tabular} \\ \midrule
Encuesta SIMCE & Agencia de Calidad & 1998 - 2014 & \begin{tabular}[c]{@{}c@{}}La encuesta SIMCE se aplica a todos los alumnos que rinden la prueba, a sus apoderados \\ y docentes. En ellos se busca capturar información de cada uno de estos actores y sus \\ percepciones frente al establecimiento y su ambiente.\end{tabular} \\ \midrule
\begin{tabular}[c]{@{}c@{}}Otros Indicadores de \\ Calidad de \\ Educación (OIC)\end{tabular} & Agencia de Calidad & 2010 - 2014 & \begin{tabular}[c]{@{}c@{}}Se presentan los resultados, para cada establecimiento, de tres de los ocho OIC: autoestima \\ académica y motivación escolar, clima de convivencia escolar, y participación y formación \\ ciudadana.\end{tabular} \\ \midrule
\begin{tabular}[c]{@{}c@{}}Información Territorial\\  por Isocrona\end{tabular} & CIT & 2013 & \begin{tabular}[c]{@{}c@{}}En estas bases de datos se presenta información geolocalizada de grupo socioeconómico, \\ niveles de delincuencia, cantidades de áreas verdes, cultura, usos de suelo, entre otros. \\ Cuando se habla de isocrona se refiere a la construcción de polígonos que representan \\ 5, 10 o 15 minutos a pie, en automóvil o en transporte público.\end{tabular} \\ \midrule
IVE-Sinae & JUNAEB & 2010 - 2015 & El IVE-SINAE se presenta para cada ciclo educacional (básica y media) de cada comuna del país. \\ \midrule
\begin{tabular}[c]{@{}c@{}}Matrícula por \\ Establecimiento\end{tabular} & \begin{tabular}[c]{@{}c@{}}RECH (hasta 2008)\\ SIGE (desde 2009)\end{tabular} & 2004 - 2014 & \begin{tabular}[c]{@{}c@{}}En esta base de datos se presenta, para cada establecimiento, el lugar donde se encuentra, \\ el total de matrículas por género, por grado y por nivel de enseñanza (básica y media).\end{tabular} \\ \midrule
\begin{tabular}[c]{@{}c@{}}Matrícula por \\ Estudiante\end{tabular} & \begin{tabular}[c]{@{}c@{}}RECH (hasta 2008)\\ SIGE (desde 2009)\end{tabular} & 2004 - 2014 & \begin{tabular}[c]{@{}c@{}}Esta base de datos da a conocer a los estudiantes (caracterizados por un MRUN),  el establecimiento \\ al que asiste y el grado que cursa, junto a su fecha de nacimiento y el lugar de residencia.\end{tabular} \\ \midrule
\begin{tabular}[c]{@{}c@{}}Matrícula y \\ Geolocalización \\ RM\end{tabular} & CIT & 2013 - 2014 & \begin{tabular}[c]{@{}c@{}}El CIT preparó esta base de datos que contiene la geolocalización de cada alumno matriculado\\  en un establecimiento.  Si bien en Matrícula por Estudiante se encuentra el lugar de residencia \\ del alumno, en este dataset se da a conocer el posicionamiento del alumno en el sistema \\ de coordenadas geográficas (latitud y longitud).\end{tabular} \\ \midrule
\begin{tabular}[c]{@{}c@{}}Rendimiento por \\ Estudiante\end{tabular} & \begin{tabular}[c]{@{}c@{}}RECH (hasta 2008)\\ SIGE (desde 2009)\end{tabular} & 2002 - 2014 & \begin{tabular}[c]{@{}c@{}}Esta base de datos contiene, para cada alumno, el establecimiento al que asiste, el grado\\  que cursa, su género, su comuna de residencia, su promedio general anual, su porcentaje \\ de asistencia y su situación de promoción al final del año escolar (promovido, reprobado, \\ retirado, trasladado).\end{tabular} \\ \midrule
Resultados PSU & MINEDUC & 2006 - 2012 & \begin{tabular}[c]{@{}c@{}}Contiene indicadores para promedio PSU de Lenguaje, Matemática y NEM \\ por establecimiento.\end{tabular} \\ \midrule
Resultados SIMCE & Agencia de Calidad & 1998 - 2014 & \begin{tabular}[c]{@{}c@{}}Se dan a conocer los resultados tanto de las pruebas SIMCE con sus puntajes por \\ establecimiento, como de los cuestionarios SIMCE de estudiantes, docentes\\  y apoderados con sus percepciones.\end{tabular} \\ \midrule
\begin{tabular}[c]{@{}c@{}}Sistema Nacional \\ Evaluación del \\ Desempeño \\ (SNED)\end{tabular} & UNAM & 1996 - 2015 & \begin{tabular}[c]{@{}c@{}}En esta base de datos se presenta, para cada establecimiento, sus puntajes de efectividad, \\ superación, iniciativa, mejoramiento, integración e igualdad (que van entre 0 y 100), la \\ clasificación de cada uno de ellos en clústers y índice SNED. Además, indica si fueron \\ premiados con subvención (100\% o 60\%).\end{tabular} \\ \bottomrule
\end{tabular}
}
\caption{Bases de datos, origen, disponibilidad y breve descripción.}
\label{tab:datasets2}
\end{table}

\subsection{Bases de Datos}
A continuación se procede a describir las bases de datos generadas por las diferentes fuentes de información. 

En la Tabla~\ref{tab:datasets} se presentan las diferentes bases de datos, su nombre, su origen, para qué años está disponible y una breve descripción de su contenido.

\begin{table}[H]
\centering
\resizebox{\textwidth}{!}{%
\begin{tabular}{@{}|c|c|c|c|@{}}
\toprule
\textbf{Nombre} & \textbf{Fuente} & \textbf{Años Disponibles} & \textbf{Descripción} \\ \midrule
\begin{tabular}[c]{@{}c@{}}Asignación Exelencia \\ Pedagógica (AEP)\end{tabular} & CPEIP & 2002 - 2013 & \begin{tabular}[c]{@{}c@{}}Esta base de datos presenta, para cada docente, sus puntajes obtenidos en las \\ diferentes pruebas que se les aplica, su puntaje final y su estadofinal \\ de acreditación.\end{tabular} \\ \midrule
Asistencia Mensual & SIGE & 2011 - 2015 & \begin{tabular}[c]{@{}c@{}}Para cada mes de cada año, esta base de datos da a conocer los días asistidos, los días \\ de clases y el porcentaje de asistencia para cada niño (identificado con un MRUN) \\ de cada establecimiento.\end{tabular} \\ \midrule
Censo Docente & \begin{tabular}[c]{@{}c@{}}SIDOC (hasta 2010)\\ SIGE (desde 2011)\end{tabular} & 2003 - 2014 & \begin{tabular}[c]{@{}c@{}}Para cada docente (identificado con una CLAVE) se da a conocer información como \\ género,año de nacimiento,  título y si tiene 1 o más, menciones (artes plásticas, ciencias\\  naturales, lenguaje, matemática, educación física, religión, etc),  en qué tipo de institución \\ estudió, año de titulación, su función principal en el establecimiento que se encuentra\\  trabajando, el tipo de contrato, horas de trabajo, años de servicio en el establecimiento,\\  grados en los que hace clases, entre otros.\end{tabular} \\ \midrule
Dotación Docente & \begin{tabular}[c]{@{}c@{}}SIDOC (hasta 2010)\\ SIGE (desde 2011)\end{tabular} & 2003 - 2014 & \begin{tabular}[c]{@{}c@{}}En esta base de datos, se resume la dotación docente de cada establecimiento, es decir, la \\ totalidad de docentes como función principal y sus horas de contrato, total de docentes en \\ la planta directiva y sus horas de contrato, total de docentes como directores y sus horas \\ de contrato, total de inspectores y sus horas de contrato, total de orientadores y sus horas \\ de contrato, entre otros.\end{tabular} \\ \midrule
Encuesta SIMCE & Agencia de Calidad & 1998 - 2014 & \begin{tabular}[c]{@{}c@{}}La encuesta SIMCE se aplica a todos los alumnos que rinden la prueba, a sus apoderados \\ y docentes. En ellos se busca capturar información de cada uno de estos actores y sus \\ percepciones frente al establecimiento y su ambiente.\end{tabular} \\ \midrule
\begin{tabular}[c]{@{}c@{}}Otros Indicadores de \\ Calidad de \\ Educación (OIC)\end{tabular} & Agencia de Calidad & 2010 - 2014 & \begin{tabular}[c]{@{}c@{}}Se presentan los resultados, para cada establecimiento, de tres de los ocho OIC: autoestima \\ académica y motivación escolar, clima de convivencia escolar, y participación y formación \\ ciudadana.\end{tabular} \\ \midrule
\begin{tabular}[c]{@{}c@{}}Información Territorial\\  por Isocrona\end{tabular} & CIT & 2013 & \begin{tabular}[c]{@{}c@{}}En estas bases de datos se presenta información geolocalizada de grupo socioeconómico, \\ niveles de delincuencia, cantidades de áreas verdes, cultura, usos de suelo, entre otros. \\ Cuando se habla de isocrona se refiere a la construcción de polígonos que representan \\ 5, 10 o 15 minutos a pie, en automóvil o en transporte público.\end{tabular} \\ \midrule
IVE-Sinae & JUNAEB & 2010 - 2015 & El IVE-SINAE se presenta para cada ciclo educacional (básica y media) de cada comuna del país. \\ \midrule
\begin{tabular}[c]{@{}c@{}}Matrícula por \\ Establecimiento\end{tabular} & \begin{tabular}[c]{@{}c@{}}RECH (hasta 2008)\\ SIGE (desde 2009)\end{tabular} & 2004 - 2014 & \begin{tabular}[c]{@{}c@{}}En esta base de datos se presenta, para cada establecimiento, el lugar donde se encuentra, \\ el total de matrículas por género, por grado y por nivel de enseñanza (básica y media).\end{tabular} \\ \midrule
\begin{tabular}[c]{@{}c@{}}Matrícula por \\ Estudiante\end{tabular} & \begin{tabular}[c]{@{}c@{}}RECH (hasta 2008)\\ SIGE (desde 2009)\end{tabular} & 2004 - 2014 & \begin{tabular}[c]{@{}c@{}}Esta base de datos da a conocer a los estudiantes (caracterizados por un MRUN),  el establecimiento \\ al que asiste y el grado que cursa, junto a su fecha de nacimiento y el lugar de residencia.\end{tabular} \\ \midrule
\begin{tabular}[c]{@{}c@{}}Matrícula y \\ Geolocalización \\ RM\end{tabular} & CIT & 2013 - 2014 & \begin{tabular}[c]{@{}c@{}}El CIT preparó esta base de datos que contiene la geolocalización de cada alumno matriculado\\  en un establecimiento.  Si bien en Matrícula por Estudiante se encuentra el lugar de residencia \\ del alumno, en este dataset se da a conocer el posicionamiento del alumno en el sistema \\ de coordenadas geográficas (latitud y longitud).\end{tabular} \\ \midrule
\begin{tabular}[c]{@{}c@{}}Rendimiento por \\ Estudiante\end{tabular} & \begin{tabular}[c]{@{}c@{}}RECH (hasta 2008)\\ SIGE (desde 2009)\end{tabular} & 2002 - 2014 & \begin{tabular}[c]{@{}c@{}}Esta base de datos contiene, para cada alumno, el establecimiento al que asiste, el grado\\  que cursa, su género, su comuna de residencia, su promedio general anual, su porcentaje \\ de asistencia y su situación de promoción al final del año escolar (promovido, reprobado, \\ retirado, trasladado).\end{tabular} \\ \midrule
Resultados PSU & MINEDUC & 2006 - 2012 & \begin{tabular}[c]{@{}c@{}}Contiene indicadores para promedio PSU de Lenguaje, Matemática y NEM \\ por establecimiento.\end{tabular} \\ \midrule
Resultados SIMCE & Agencia de Calidad & 1998 - 2014 & \begin{tabular}[c]{@{}c@{}}Se dan a conocer los resultados tanto de las pruebas SIMCE con sus puntajes por \\ establecimiento, como de los cuestionarios SIMCE de estudiantes, docentes\\  y apoderados con sus percepciones.\end{tabular} \\ \midrule
\begin{tabular}[c]{@{}c@{}}Sistema Nacional \\ Evaluación del \\ Desempeño \\ (SNED)\end{tabular} & UNAM & 1996 - 2015 & \begin{tabular}[c]{@{}c@{}}En esta base de datos se presenta, para cada establecimiento, sus puntajes de efectividad, \\ superación, iniciativa, mejoramiento, integración e igualdad (que van entre 0 y 100), la \\ clasificación de cada uno de ellos en clústers y índice SNED. Además, indica si fueron \\ premiados con subvención (100\% o 60\%).\end{tabular} \\ \bottomrule
\end{tabular}
}
\caption{Bases de datos, origen, disponibilidad y breve descripción.}
\label{tab:datasets}
\end{table}
