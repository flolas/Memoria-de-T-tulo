\subsubsection{CIAE}
El CIAE igualmente nos entregó conjuntos de datos útiles para la creación de los indicadores, como por ejemplo información de las distancias de los alumnos a los establecimientos y los llamados profesores taxi, todo esto utilizando información del CIT.
\begin{longdescription}
  \item[Distancias a Establecimiento] \hfill \\
  Continene información sobre la distancia de los establecimientos a las diferentes manzanas de Santiago.
    \begin{longdescription}
        \item[Distancia del Alumno al Establecimiento mas Cercano](DIST\_ALU\_A\_RBD\_C): Cálculo de la distancia euclidiana desde la manzana del establecimiento más cercano al alumno $j$ hasta su  manzana de residencia.
        \item[Distancia del Alumno al Establecimiento Autonomo\footnote{[cita] explicar autónomo} mas Cercano](DIST\_ALU\_A\_RBD): Cálculo de la distancia euclidiana desde la manzana del establecimiento autonomo más cercano al alumno $j$ hasta su  manzana de residencia.
        \item[Distancia del Alumno al Establecimiento que asiste](DIST\_ALU\_A\_RBD\_AUT\_C):
        Cálculo de la distancia euclidiana desde la manzana del establecimiento que asiste el alumno $j$ hasta su  manzana de residencia.
    \end{longdescription}
  \item[Profesores Taxi] \hfill \\
  Continene información sobre los profesores taxi de los establecimientos de Chile
    \begin{longdescription}
        \item[Cantidad Promedio de Establecimientos donde trabajan los Profesores](PROM\_CANT\_ESTAB\_DIF\_DOC\_RBD): La cantidad promedio de establecimientos diferentes donde trabajan los docentes del establecimiento que asiste el alumno $j$.
        \item[Cantidad Promedio de Establecimientos de diferente región donde trabajan los Profesores](PROM\_CANT\_ESTAB\_DIF\_REG\_DOC\_RBD):La cantidad promedio de establecimientos de diferentes regiones donde trabajan los docentes del establecimiento que asiste el alumno $j$.
    \end{longdescription}    
\end{longdescription}