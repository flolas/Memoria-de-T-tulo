\subsubsection{Agencia de Calidad}
A continuación se presentan los indicadores creados a partir de los conjuntos de datos recibidos por parte de la Agencia de Calidad.\footnote{Esta información se obtuvo a través de la Ley de Transparencia y la URL \url{http://www.agenciaeducacion.cl}}

Como fue mencionado en capítulos anteriores, la Agencia de Calidad es la encargada de las pruebas SIMCE, aplicadas a estudiantes, docentes y apoderados. Es de alta importancia mencionar que de estos tres, sólo se utilizarán las encuestas de los apoderados debido a que los resultados que arroja la encuesta a los estudiantes no permite hacer una comparación entre ellos, sino que entre establecimientos. Y como la finalidad es realizar un sistema que permita detectar de forma preventiva a los alumnos que desertarán, utilizar estos resultados con agregación a nivel de establecimiento se pierde la granularidad necesaria. Con respecto a la encuesta realizada a los docente, los resultados no son relevantes para el objetivo, y es por eso que tampoco se utilizan. 

\begin{longdescription}
  \item[Encuesta SIMCE] \hfill \\
    Como hemos mencionado, la encuesta SIMCE mide diferentes aristas sobre el rendimiento de los establecimientos y alumnos, pero también contiene encuestas de utilidad que son realizadas por los padres. Lamentablemente no fue posible acceder a la información de las encuestas de todos los años, puesto que la Agencia de Calidad entregó la información con llave únicas duplicadas, e incluso más de una vez.
     \begin{longdescription}
        \item[Educación Media y Básica del Padre](EDU\_P): Describe la escolaridad de la educación básica y media del Padre del alumno.
        \item[Educación Media y Básica de la Madre](EDU\_M): Describe la escolaridad de la educación básica y media de la Madre del alumno.
        \item[Educación Superior del Padre](EDU\_SUP\_P): Señala si el Padre del alumno cursó estudios superiores y los finalizó.
        \item[Educación Superior de la Madre](EDU\_SUP\_M): Señala si la Madre del alumno cursó estudios superiores y los finalizó.
        \item[Ingreso del Hogar del Alumno](ING\_HOGAR): Describe el ingreso del hogar por tramos del lugar donde reside el alumno.
     \end{longdescription}
  \item[Otros Indicadores de la Calidad] \hfill \\
    Como ya se ha hablado en la sección anterior, los OIC serán los futuros indicadores de la Agencia de Calidad para controlar la salud de la educación chilena. Solamente se obtuvieron tres de parte de la Agencia de Calidad y no están disponible para todos los alumnos.\footnote{Metodología de la creación de los OIC[cita]}
       \begin{longdescription}
        \item[Indicador Promedio de Convivencia Escolar](CONVIVENCIA\_PROM\_RBD): El cálculo se realiza obteniendo el promedio del indicador de todos los alumnos del establecimiento que tienen disponible el indicador, la metodología de cálculo no se logró conseguir.
        \item[Indicador Promedio de Autoestima Escolar](AUTOESTIMA\_MOTIVACION\_PROM\_RBD): El cálculo se realiza obteniendo el promedio del indicador de todos los alumnos del establecimiento que tienen disponible el indicador, la metodología de cálculo no se logró conseguir.
        \item[Indicador Promedio de Participación Escolar](PARTICIPACION\_PROM\_RBD): El cálculo se realiza obteniendo el promedio del indicador de todos los alumnos del establecimiento que tienen disponible el indicador, la metodología de cálculo no se logró conseguir.
       \end{longdescription}
\end{longdescription}