\section{Preparación de los Datos}
Como se señaló en el Capítulo~\ref{ch:metodologia}, la preparación de los datos es la etapa más importante para la minería de datos. Durante este ca
\subsection{Creación de Indicadores}
En base a la revisión teórica en el Capitulo~\ref{ch:marcoteorico} se encontraron diferentes indicadores tentativos que son factibles para su creación.
A continuación presentaremos la creación de cada uno de estos en base a la entidad de dónde provienen. Además, todos los indicadores están relacionados a un alumno, en este caso, nuestras observaciones consisten en alumnos y nuestros campos o columnas, corresponden a los indicadores o características del alumno.
\cindicadores{MINEDUC}
\cindicadores{AgenciaDeCalidad}
\cindicadores{MIME}
\cindicadores{JUNAEB}
\cindicadores{CIAE}
\cindicadores{CIT}
\subsection{Preparación de la Muestra de Estudio}
\subsection{Validación de la Muestra de Estudio}
\subsection{Pre-procesamiento de la Muestra de Estudio}



