\section{Creación de Indicadores}
En base a la información contenida en los \textit{datasets} disponibles, se procede a generar una lista de indicadores. 

La finalidad de crear estos indicadores radica en poder resumir información para luego poder utilizarla en el modelo de predicción. Esto resulta útil ya que, al ser alta la cantidad de información disponible, es posible optimizar los tiempos de procesamiento, disminuyendo el número de variables de entrada. 

Los indicadores creados resumen información de tres esferas: características personales del alumno, características intraescolares del establecimiento al que asiste el alumno, e información relativa al ambiente familiar del alumno. 

A continuación, se procede a detallar estos indicadores.

\subsection{Indicadores Personales del Alumno}
Los indicadores de esta índole dan a conocer información con respecto al alumno y ...

\begin{itemize}
\item Sobre-edad: representa la diferencia entre la edad de un alumno y la edad esperada para el nivel que se encuentra cursando.
\item Alumno repitente: indica si el año anterior, el alumno cursaba el mismo grado que cursa un año dado.
\item Comuna de residencia y estudio: da a conocer si el alumno estudia o no en la misma comuna en la que vive.
\item Curso del alumno: indica el grado que cursa el alumno.
\item Género del alumno: indica el género del alumno.
\item Distancia residencia-establecimiento: da a conocer la distancia en kms. de la residencia del alumno al establecimiento educacional al que asiste.
\item Número de traslado al año: indica el número de traslados de un alumno en un año escolar. 
\item Rendimiento:
\item Asistencia: 
\end{itemize}


\subsection{Indicadores Intraescolares}

Por otro lado, se aborda lo que se relaciona con el contexto intraescolar, como la cantidad promedio de alumnos por curso, y aquello relacionado con los docentes del establecimiento, como el promedio de horas lectivas\footnote{Ver Anexo.}.

\subsection{Indicadores Familiares}

Finalmente, estos indicadores capturan también información sobre el ambiente extraescolar y de la familia de cada alumno\footnote{Ver Anexo.}. 

\begin{itemize}
\item Grupo socioeconómico:
\item 
\end{itemize}